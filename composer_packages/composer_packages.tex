\documentclass[pdf]{beamer}
\usetheme{Warsaw}
\definecolor{links}{HTML}{2A1B81}
\hypersetup{colorlinks,linkcolor=,urlcolor=links}
\usepackage{xcolor}%
% Listings config CC-BY-SA CJ Harries 2015
% JSON config http://tex.stackexchange.com/questions/83085/how-to-improve-listings-display-of-json-files
\usepackage{listings}%
\colorlet{punct}{red!60!black}%
\definecolor{background}{HTML}{EEEEEE}%
\definecolor{delim}{RGB}{20,105,176}%
\colorlet{numb}{magenta!60!black}%
\lstdefinelanguage{json}{
    literate=
     *{0}{{{\color{numb}0}}}{1}
      {1}{{{\color{numb}1}}}{1}
      {2}{{{\color{numb}2}}}{1}
      {3}{{{\color{numb}3}}}{1}
      {4}{{{\color{numb}4}}}{1}
      {5}{{{\color{numb}5}}}{1}
      {6}{{{\color{numb}6}}}{1}
      {7}{{{\color{numb}7}}}{1}
      {8}{{{\color{numb}8}}}{1}
      {9}{{{\color{numb}9}}}{1}
      {:}{{{\color{punct}{:}}}}{1}
      {,}{{{\color{punct}{,}}}}{1}
      {\{}{{{\color{delim}{\{}}}}{1}
      {\}}{{{\color{delim}{\}}}}}{1}
      {[}{{{\color{delim}{[}}}}{1}
      {]}{{{\color{delim}{]}}}}{1},
}
\lstset{
	% Basic design
    basicstyle={\small\ttfamily},   
    frame=l,
    % Line numbers
    xleftmargin={0.75cm},
    numbers=left,
    stepnumber=1,
    firstnumber=1,
    numberfirstline=true,
    tabsize=2,
    showtabs=false,
    showspaces=false,
    showstringspaces=false,
    extendedchars=true,
    breaklines=true, 
}
\newcommand{\com}[1]{\lstinline$#1$}

\mode<presentation>{}
\title{Composer Packages}
\author{CJ Harries}
\date{1/19/15}

\begin{document}
\begin{frame}
\titlepage
\end{frame}

\section{Introduction}
\subsection{What is Composer?}
\begin{frame}
\href{https://getcomposer.org/}{Composer} is a dependency manager for PHP. It allows developers to automatically load and use external code without manually moving and updating files. It is the core of many PHP frameworks, including Laravel.
\end{frame}

\subsection{How do I install Composer?}
\begin{frame}[fragile]
With curl:
\begin{lstlisting}[language=bash]
curl -sS https://getcomposer.org/installer | php
\end{lstlisting}
Without curl:
\begin{lstlisting}[language=bash]
php -r "readfile('https://getcomposer.org/installer');" | php
\end{lstlisting}
Options are \href{https://getcomposer.org/download/}{here}.
\end{frame}

\subsection{How do I make Composer easy to use?}
\begin{frame}[fragile]
With decent operating systems:
\begin{lstlisting}[language=bash]
mv composer.phar /usr/local/bin/composer
\end{lstlisting}
(Note: on some versions of OSX you might also need to run \com{mkdir -p /usr/local/bin} before the above will work)

Without decent operating systems:
\begin{lstlisting}
google.com
\end{lstlisting}
\end{frame}

\subsection{How do I use Composer?}
\begin{frame}[fragile]
Easy way:
\begin{lstlisting}[language=bash]
composer --help
\end{lstlisting}
Hard way:
\begin{lstlisting}[language=bash]
php composer.phar --help
\end{lstlisting}
\end{frame}

\section{\com{composer.json}}
\subsection{What is \com{composer.json}?}
\begin{frame}
\com{composer.json} contains all of the information your project needs to describe itself and load dependencies. It's similar to RequireJS's \com{main.js} or Android's Gradle system. Your project needs one, your library needs one, and your external dependencies come with one.

It lives in the root directory. If it's not there, Composer can't find it.
\end{frame}

\subsection{What does \com{composer.json} look like?}
\begin{frame}[fragile]
\begin{lstlisting}[language=json]
{
  "name": "sntai/core",
  "version": "0.0.00",
  "description": "Core AI functionality",
  "type": "library",
  "license": "Proprietary-SNT",
  "keywords": ["ai"],
  "authors": [
    {
      "name": "CJ Harries",
      "email": "cjharries@sntmedia.com"
    }
  ],
  "autoload": {
    "psr-4": {
      "SNTAI\\Core\\" : "src"
    }
  }
}
\end{lstlisting}
\end{frame}
\begin{frame}[fragile]
\begin{lstlisting}[language=json]
{
  "name": "sntai/apifin",
  "version": "0.0.00",
  "description": "API AI FIN",
  "type": "library",
  "license": "Proprietary-SNT",
  "keywords": ["ai"],
  "repositories": [
    {
      "type": "vcs",
      "url": "https://github.com/passfail-cjharries/sntai_core"
    }
  ],
  "require": {
    "sntai/core": "dev-master"
  }
}
\end{lstlisting}
\end{frame}

\subsection{What can I do with \com{composer.json}?}
\begin{frame}
Like any configuration file, \com{composer.json} gives you the option to define
\begin{itemize}
\item Project information (author, version, license, etc.)
\item Exports and configuration (autoloading, standards, unit tests, etc.)
\item Dependencies (other composer packages, VCS, local files, etc.)
\end{itemize}
\end{frame}

\subsection{Why do I want to make a \com{composer.json}?}
\begin{frame}
\textbf{You don't have to ctrl+c/ctrl+v code.}
\end{frame}

\section{Making a Composer package}
\subsection{What should a Composer package look like?}
\begin{frame}[fragile]
Make a new directory for your project and organize your files in a way that makes sense, with a \com{composer.json} in the root directory.
\begin{lstlisting}
/sntai_core
	/src
		AiCache.php
	/views
		SomeHtml.html
	/tests
		testAiCache.php
composer.json
\end{lstlisting}
\end{frame}
\subsection{How do I load the files?}
\begin{frame}[fragile]
Add source and tests to \com{composer.json} (along with any other exports)
\begin{lstlisting}[language=json]
{
  ...
  "autoload": {
    "psr-4": { "SNTAI\\Core\\": "src" }
  },
  "autoload-dev": {
    "psr-4: { "SNTAIT\\Core\\Tests\\" : "tests" }
  },
  ...
}
\end{lstlisting}
\end{frame}
\subsection{How do does Composer know what my package is?}
\begin{frame}[fragile]
Name and version (if you like that sort of thing) it
\begin{lstlisting}[language=json]
{
  ...
  "name": "sntai/core",
  "version": "0.0.01",
  ...
}
\end{lstlisting}
\end{frame}
\subsection{Can I name it anything?}
\begin{frame}[fragile]
(Optional) Alias the branches you want to create through \com{git}
\begin{lstlisting}[language=json]
{
  ...
  "extra": {
    "branch-alias": {
      "prod": "development"
    }
  },
  ...
}
\end{lstlisting}
\end{frame}
\subsection{Where should I put my package?}
\begin{frame}[fragile]
Do the \com{git} thing
\begin{lstlisting}[language=bash]
~/sntai_core$ git init
~/sntai_core$ git remote add origin https://github.com/passit/sntai_core.git
~/sntai_core$ git push -u origin master
\end{lstlisting}
You can also add your package to \href{https://packagist.org/}{Packagist}, but you probably shouldn't do that with proprietary code.
\end{frame}

\section{Linking Composer packages}
\subsection{How do I convert an existing project?}
\begin{frame}[fragile]
Navigate to the root directory and make a new \com{composer.json} with any package information you want to add.
\begin{lstlisting}
// You've seen this before
\end{lstlisting}
\end{frame}
\begin{frame}[fragile]
You'll probably want to update your \com{.gitignore} to add the \com{vendor} directory. If you don't, you'll pollute your repos with external code.
\begin{lstlisting}
vendor
/vendor/
vendor/*
# You have the power
\end{lstlisting}
\end{frame}

\subsection{How do I link my other packages?}
\begin{frame}
Composer is incredibly powerful in that it can pull from Packagist, local sources, VCS, and basically anything \href{https://getcomposer.org/doc/02-libraries.md}{you try to}. Composer can even \href{https://getcomposer.org/doc/articles/http-basic-authentication.md}{connect to your own private repos} with some fiddling.
\end{frame}
\begin{frame}[fragile]
\begin{description}
\item[Packagist] is the easiest to use.
\end{description}
\begin{lstlisting}[language=json]
{
  ...
  "require": {
    "monolog/monolog": "1.0.*"
  },
  ...
}
\end{lstlisting}
\end{frame}
\begin{frame}[fragile]
\begin{description}
\item[git] isn't too hard either. Notice the \com{dev-} in \com{dev-master}.
\end{description}
\begin{lstlisting}[language=json]
{
  ...
  "repositories": [
    {
      "type": "vcs",
      "url": "https://github.com/passit/sntai_core"
    }
  ],
  "require": {
    "sntai/core": "dev-master"
  },
  ...
}
\end{lstlisting}
\end{frame}
\begin{frame}[fragile]
\begin{description}
\item[Private repos] work as well (with some tweaking and potentially some ssh-fu). Notice the \com{dev-} in \com{dev-master}.
\end{description}
\begin{lstlisting}[language=json]
{
  ...
  "repositories": [
    {
      "type": "vcs",
      "url": "git@github.com:passit/sntai_core.git"
    }
  ],
  "require": {
    "sntai/core": "dev-master"
  },
  ...
}
\end{lstlisting}
\end{frame}
\subsection{How do I update dependencies?}
\begin{frame}[fragile]
Navigate to the root directory of your project.

First time:
\begin{lstlisting}[language=bash]
~/sntai_core$ composer install
\end{lstlisting}
Any other time:
\begin{lstlisting}[language=bash]
~/sntai_core$ composer update
\end{lstlisting}
Hard way:
\begin{lstlisting}[language=bash]
~/sntai_core$ php composer.phar install/update
\end{lstlisting}
\end{frame}
\section{Other stuff}
\subsection{What's next?}
\begin{frame}
\begin{itemize}
\item Using Composer \href{https://getcomposer.org/doc/03-cli.md}{from the command line} is very powerful.
\item The \href{https://getcomposer.org/doc/04-schema.md}{\com{composer.json} schema} has tons of options.
\item \href{https://www.google.com/webhp?q=composer+tricks}{The internet} has some nice things to say about Composer.
\end{itemize}
\end{frame}
\subsection{Sources}
\begin{frame}
\begin{itemize}
\item \href{https://getcomposer.org/doc/}{Composer docs}
\item \href{https://stackoverflow.com}{StackOverflow}
\item \href{https://google.com}{Google}
\end{itemize}
\end{frame}
\end{document}